% !TeX encoding = UTF-8
\section{Introduction}
\textit{The graph coloring problem} is one of the most famous \textit{NP-complete} problems. Given a graph $G = (V, E)$ and $k$, where $V$ is the set of vertices in the graph, $E$ is the set of edges and $k$ is the number of available colors. The problem is to assign a certain color to every vertex $v \in V$ with the constraint: there is no pair of vertices which are adjacent and  have the same color.
This problem has many applications:
\begin{itemize}
    \item[(1)] \textit{Map Coloring:} Coloring geographical maps of countries or states where no two adjacent countries can be assigned same color. Four colors are enough to color.
    \item[(2)] \textit{Sudoku:} Sudoku is a variation of graph coloring problem where every cell represents a vertex. There is an edge between two vertices if they are in same row or same column or same block.
    \item[(3)] \textit{Register Allocation:} In compiler optimization, register allocation is the process of assigning a large number of target program variables onto a small number of CPU registers. \cite{example}
\end{itemize}

\begin{definition}
A \textit{graph} is a tuple $G = (V, E)$ consisting of a nonempty set $V$ of vertices and a set of edges $E$. \cite{bondy1976graph}
\end{definition}

\begin{definition}
A \textit{vertex coloring} of a graph $G = (V, E)$ is a map $c: V \longrightarrow S$ such that $c(v) \ne c(w)$ whenever $v$ and $w$ are adjacent. The elements of the set $S$ are called the available colors. If $G$ has a $k$-COLORING, namely the size of $S$ is $k$, then we call the graph $G$ is $k$-colorable.
 \cite{diestel2010graph}
\end{definition}
\begin{claim}
2-COLORING problem can be solved in polynomial time.
\end{claim}

\begin{proof}
Suppose the given graph $G = (V, E)$ and color $c_1, c_2$. 
\begin{itemize}
    \item[(1)] Randomly pick $v \in V$, color it with $c_1$.
    \item[(2)] Apply BFS starting with vertex $v$ and color its neighbors with $c_2$. The point is that for $\forall u \in V$, we should color its neighbors with the other color alternatively: assume that the color of $u$ is $c_1$, then we assign $c_2$ to its neighbors, and vice versa.
    \item[(3)] After finishing BFS, go through all vertices again to check whether there is a vertex that is assigned with same colors other than its neighbor(s). If \textit{yes}, then the graph is \textit{not} 2-colorable, \textit{otherwise} it is 2-colorable.
\end{itemize}
The running time is similar to BFS: $\mathcal{O}(|V| + |E|)$.
\end{proof}

\begin{observation}
We can check whether a graph is bipartite by coloring the graph using two colors. If a given graph is 2-colorable, then it is bipartite.
\end{observation}

\begin{observation}
$k$-COLORING problems are $NP-complete$ for $k \geq 3$. 
\end{observation}

As previously stated, the k-COLORING problem is NP-complete, which means that it seems hardly possible to have a polynomial time algorithm for this problem. Moving on now to consider 3-COLORING problem under circumstance that there is no triangle within the given graph $G$. Grötzsch’s theorem \cite{grotzsch1959dreifarbensatz} states that each triangle-free planar graph is 3-colorable. Thomassen \cite{Thomassen1994Grtzschs3T} has also found two proofs and extended the result in various way, by which a quadratic algorithm for finding suitable 3-coloring can be developed possibly. Kowalik \cite{article} maintains a complex data structure called \textit{Short Path Data Structure (SPDS)}, which will be built in linear time and enables that finding shortest paths of length at most 2 in planar graph takes $\mathcal{O}(1)$ time. The SPDS will be constantly updated during the particular sequence of operations. And its running time for finding 3-COLORING is $\mathcal{O}(n\log{}n)$. After that, Dvorak, Kawaravayashi and Thomas \cite{dvorak2013threecoloring} have designed a linear-time algorithm which still relies on the Grötzsch’s theorem but avoids complex data structures. Nevertheless, their paper is quite complicated for readers. So this thesis will give a deeper and detailed view into their paper for better understanding of their main ideas and proofs. \\ 

In the second section, we will give some needed definitions and give an overview of this linear-time algorithm. In the third section, we will introduce a attribute called safety with which it's possible to reduce the size of graph. Then in the fourth section, we will give a short proof of Grötzsch's theorem by two lemmas. Next, we will give an overview how a naive algorithm for three-coloring is implemented and show some improvements. In the last two sections, the way to have a linear-time algorithm will be given and its correctness will be also proven. 
